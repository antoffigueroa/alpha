Heckman et al. 1990: \\
\citet{1990ApJS...74..833H} \\
\citep{1990ApJS...74..833H} \\
observational evidence of outflows along the minor axis of galaxies in the local
universe. \\

\hline

Lehnert and Heckman 1996: \\
\citet{1996ApJ...462..651L} \\
\citep{1996ApJ...462..651L} \\
excess of line emission along the minor axis of galaxies in the local Universe.
Lines are broader along the minor axis. \\

\hline

Bordoloi et al. 2011: \\
\citet{2011ApJ...743...10B} \\
\citep{2011ApJ...743...10B} \\
studied azimuthal distribution of Mg \textsc{ii} at <200 kpc of $\sim$4000
galaxies at 0.5 < $z$ < 0.9. Found strong azimuthal dependence of Mg \textsc{ii}
absorption. \\

\hline

Bouche et al 2012b: \\
\citet{2012MNRAS.426..801B} \\
\citep{2012MNRAS.426..801B} \\
studied strong Mg \textsc{ii} absorption system in the line of sight of 12 QSOs.
Azimuthal distribution is bimodal. Biconical outflow model is able to reproduce
the results. \\

\hline

Dijkstra and Kramer 2012: \\
\citet{2012MNRAS.424.1672D} \\
\citep{2012MNRAS.424.1672D} \\
simulations of the CGM that take into consideration line strength dependence on
azimuthal angle. \\

\hline

Gauthier and Chen 2012: \\
\citet{2012MNRAS.424.1952G} \\
\citep{2012MNRAS.424.1952G} \\
studied outflow properties of four galaxies at $z$ > 0.5. Created an outflow
model to constrain outflow apperture angles. \\

\hline

Kacprzak et al. 2012: \\
\citet{2012ApJ...760L...7K} \\
\citep{2012ApJ...760L...7K} \\
studied 88 Mg \textsc{ii} absorption systems and 35 non-absorption systems.
Found absorption is preferently located along the minor and major axes. \\

\hline

Kornei et al. 2012: \\
\citet{2012ApJ...758..135K} \\
\citep{2012ApJ...758..135K} \\
studied outflows on 72 galaxies at $z\sim$1. Found outflows are not spherically
distributed. \\

\hline

Lundgren et al. 2012: \\
\citet{2012ApJ...760...49L} \\
\citep{2012ApJ...760...49L} \\
azimuthal dependence of Mg \textsc{ii} equivalent width seems to be weaker at
higher z, indicating a possible collimation of galaxy scale outflows with time.
\\

\hline

Martin et al. 2012: \\
\citet{2012ApJ...760..127M} \\
\citep{2012ApJ...760..127M} \\
studied $\sim$200 galaxies at 0.4 < $z$ < 1.4. Found a strong correlation
between Fe \textsc{ii} velocity and azimuthal angle. \\

\hline

Bouche et al. 2013: \\
\citet{2013Sci...341...50B} \\
\citep{2013Sci...341...50B} \\
inflows are located along the minor axis. \\

\hline

Zhu and Ménard 2013: \\
\citet{2013ApJ...773...16Z} \\
\citep{2013ApJ...773...16Z} \\
studied $\sim$10$^{6}$ absorption systems using Ca \textsc{ii} absorption lines
in background QSOs. Ca \textsc{ii} seems to be preferently located along the
minor axis of galaxies. \\

\hline

Bordoloi et al. 2014a: \\
\citet{2014ApJ...784..108B} \\
\citep{2014ApJ...784..108B} \\
studied Mg \textsc{ii} absorption using QSO absorption lines and stacked
spectra. Found a strong correlation between line strength and azimuthal angle.
Stronger absorbers are within 50º of the minor axis. \\

\hline

Bordoloi et al. 2014b: \\
\citet{2014ApJ...794..130B} \\
\citep{2014ApJ...794..130B} \\
studied 486 galaxies at 1 < $z$ < 1.5. Found that face-on galaxies show higher
equivalent widths than edge-on galaxies. \\

\hline

Kacprzak et al. 2014: \\
\citet{2014ApJ...792L..12K} \\
\citep{2014ApJ...792L..12K} \\
found evidence for an outflow along the minor axis of a galaxy at $z$=0.2. \\

\hline

Lan et al. 2014: \\
\citet{2014ApJ...795...31L} \\
\citep{2014ApJ...795...31L} \\
studied 2000 galaxy-absorber pairs at $z\sim$0.5. Found stronger
Mg \textsc{ii} absorbers are located along the minor axis. \\

\hline

Mathes et al. 2014: \\
\citet{2014ApJ...792..128M} \\
\citep{2014ApJ...792..128M} \\
studied 14 H \textsc{i} and O \textsc{vi} absorption systems. Found
H \textsc{i} column density is larger along the minor and major axes of
galaxies, but O \textsc{vi} seemed to be uniform. \\

\hline

Rubin et al. 2014: \\
\citet{2014ApJ...794..156R} \\
\citep{2014ApJ...794..156R} \\
studied Mg \textsc{ii} and Fe \textsc{ii} absorption profiles on 105 galaxies at
0.3 < $z$ < 1.4. Found constraints on the opening angle of outflows. \\

\hline

Kacprzak et al. 2015: \\
\citet{2015MNRAS.446.2861K} \\
\citep{2015MNRAS.446.2861K} \\
studied Mg \textsc{ii} absorption on 14 AGNs at 0.12 < $z$ < 0.22. Found no
dependence of Mg \textsc{ii} $C_{F}$ on azimuthal angle, although they
acknowledge there could be a hint of one\\

\hline

Schroetter et al. 2015: \\
\citet{2015ApJ...804...83S} \\
\citep{2015ApJ...804...83S} \\
observations of 10 QSO-galaxy pairs. Model that assumes a biconical outflow
along the minor axis is able to explain the observations. \\

\hline

Schroetter et al. 2016: \\
\citet{2016ApJ...833...39S} \\
\citep{2016ApJ...833...39S} \\
found evidence for a biconical outflow in a single galaxy using MUSE and
UVES. \\

\hline

Kacprzak 2017: \\
\citet{2017ASSL..430..145K} \\
\citep{2017ASSL..430..145K} \\
review. Relevant sections: 2.2, 4. \\

\hline

Nelson et al. 2019: \\
\citet{2019MNRAS.490.3234N} \\
\citep{2019MNRAS.490.3234N} \\
simulations of outflows. Initially outflows have a wide apperture angle, but
then become collimated. \\

\hline

Schroetter et al. 2019: \\
\citet{2019MNRAS.490.4368S} \\
\citep{2019MNRAS.490.4368S} \\
studied 79 strong Mg \textsc{ii} absorbers. Found bimodal distribution of
Mg \textsc{ii} equivalent width. \\

\hline

Lundgren et al. 2021: \\
\citet{2021ApJ...913...50L} \\
\citep{2021ApJ...913...50L} \\
studied 54 Mg \textsc{ii} absorbers at 0.64 < $z$ < 1.4. Found enhanced
Mg \textsc{ii} absorption within 50º of the minor axis. Also calculated a model
for equivalent width distribution that takes into consideration azimuthal angle,
impact parameter and environment.
