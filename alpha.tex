Heckman et al. 1990: \\
\citet{1990ApJS...74..833H} \\
\citep{1990ApJS...74..833H} \\
observational evidence of outflows along the minor axis of galaxies in the local
universe. \\

\hline

Hjelm and Lindblad 1996: \\
\citet{1996A&A...305..727H} \\
\citep{1996A&A...305..727H} \\
studied the morphology and kinematics of a single galaxy. Found a biconical
outflow with an opening angle of 100\º. \\

\hline

Lehnert and Heckman 1996: \\
\citet{1996ApJ...462..651L} \\
\citep{1996ApJ...462..651L} \\
excess of line emission along the minor axis of galaxies in the local Universe.
Lines are broader along the minor axis. \\

\hline

Veilleux et al. 2001: \\
\citet{2001AJ....121..198V} \\
\citep{2001AJ....121..198V} \\
studied the outflow of a single galaxy. Found its apperture angle is $\sim$
125\º-135\º. \\

\hline

Walter et al. 2002: \\
\citet{2002ApJ...580L..21W} \\
\citep{2002ApJ...580L..21W} \\
studied the CGM of a single galaxy. Found a biconical outflow with an opening
angle of $\sim$55\º. \\

\hline

Bordoloi et al. 2011: \\
\citet{2011ApJ...743...10B} \\
\citep{2011ApJ...743...10B} \\
studied azimuthal distribution of Mg \textsc{ii} at <200 kpc of $\sim$4000
galaxies at 0.5 < $z$ < 0.9. Found strong azimuthal dependence of Mg \textsc{ii}
absorption. \\

\hline

Bouche et al 2012b: \\
\citet{2012MNRAS.426..801B} \\
\citep{2012MNRAS.426..801B} \\
studied strong Mg \textsc{ii} absorption system in the line of sight of 12 QSOs.
Azimuthal distribution is bimodal. Biconical outflow model is able to reproduce
the results. \\

\hline

Dijkstra and Kramer 2012: \\
\citet{2012MNRAS.424.1672D} \\
\citep{2012MNRAS.424.1672D} \\
simulations of the CGM that take into consideration line strength dependence on
azimuthal angle. \\

\hline

Gauthier and Chen 2012: \\
\citet{2012MNRAS.424.1952G} \\
\citep{2012MNRAS.424.1952G} \\
studied outflow properties of four galaxies at $z$ > 0.5. Created an outflow
model to constrain outflow apperture angles. \\

\hline

Kacprzak et al. 2012: \\
\citet{2012ApJ...760L...7K} \\
\citep{2012ApJ...760L...7K} \\
studied 88 Mg \textsc{ii} absorption systems and 35 non-absorption systems.
Found absorption is preferently located along the minor and major axes. \\

\hline

Kornei et al. 2012: \\
\citet{2012ApJ...758..135K} \\
\citep{2012ApJ...758..135K} \\
studied outflows on 72 galaxies at $z\sim$1. Found outflows are not spherically
distributed. \\

\hline

Lundgren et al. 2012: \\
\citet{2012ApJ...760...49L} \\
\citep{2012ApJ...760...49L} \\
azimuthal dependence of Mg \textsc{ii} equivalent width seems to be weaker at
higher z, indicating a possible collimation of galaxy scale outflows with time.
\\

\hline

Martin et al. 2012: \\
\citet{2012ApJ...760..127M} \\
\citep{2012ApJ...760..127M} \\
studied $\sim$200 galaxies at 0.4 < $z$ < 1.4. Found a strong correlation
between Fe \textsc{ii} velocity and azimuthal angle. \\

\hline

Bouche et al. 2013: \\
\citet{2013Sci...341...50B} \\
\citep{2013Sci...341...50B} \\
inflows are located along the minor axis. \\

\hline

Zhu and Ménard 2013: \\
\citet{2013ApJ...773...16Z} \\
\citep{2013ApJ...773...16Z} \\
studied $\sim$10$^{6}$ absorption systems using Ca \textsc{ii} absorption lines
in background QSOs. Ca \textsc{ii} seems to be preferently located along the
minor axis of galaxies. \\

\hline

Bordoloi et al. 2014a: \\
\citet{2014ApJ...784..108B} \\
\citep{2014ApJ...784..108B} \\
studied Mg \textsc{ii} absorption using QSO absorption lines and stacked
spectra. Found a strong correlation between line strength and azimuthal angle.
Stronger absorbers are within 50º of the minor axis. \\

\hline

Bordoloi et al. 2014b: \\
\citet{2014ApJ...794..130B} \\
\citep{2014ApJ...794..130B} \\
studied 486 galaxies at 1 < $z$ < 1.5. Found that face-on galaxies show higher
equivalent widths than edge-on galaxies. \\

\hline

Kacprzak et al. 2014: \\
\citet{2014ApJ...792L..12K} \\
\citep{2014ApJ...792L..12K} \\
found evidence for an outflow along the minor axis of a galaxy at $z$=0.2. \\

\hline

Lan et al. 2014: \\
\citet{2014ApJ...795...31L} \\
\citep{2014ApJ...795...31L} \\
studied 2000 galaxy-absorber pairs at $z\sim$0.5. Found stronger
Mg \textsc{ii} absorbers are located along the minor axis. \\

\hline

Mathes et al. 2014: \\
\citet{2014ApJ...792..128M} \\
\citep{2014ApJ...792..128M} \\
studied 14 H \textsc{i} and O \textsc{vi} absorption systems. Found
H \textsc{i} column density is larger along the minor and major axes of
galaxies, but O \textsc{vi} seemed to be uniform. \\

\hline

Rubin et al. 2014: \\
\citet{2014ApJ...794..156R} \\
\citep{2014ApJ...794..156R} \\
studied Mg \textsc{ii} and Fe \textsc{ii} absorption profiles on 105 galaxies at
0.3 < $z$ < 1.4. Found constraints on the opening angle of outflows. \\

\hline

Borthakur et al. 2015: \\
\citet{2015ApJ...813...46B} \\
\citep{2015ApJ...813...46B} \\
studied the CGM of 45 low-z galaxies at large impact parameters. Found no
dependence of Ly-$\alpha$ equivalent width with azimuthal angle. Attributed
this result to the large impact parameters. \\

\hline

Kacprzak et al. 2015a: \\
\citet{2015MNRAS.446.2861K} \\
\citep{2015MNRAS.446.2861K} \\
studied Mg \textsc{ii} absorption on 14 AGNs at 0.12 < $z$ < 0.22. Found no
dependence of Mg \textsc{ii} $C_{F}$ on azimuthal angle, although they
acknowledge there could be a hint of one\\

\hline

Kacprzak et al. 2015b: \\
\citet{2015ApJ...815...22K} \\
\citep{2015ApJ...815...22K} \\
found that O \textsc{vi} equivalent width is higher on absorbers located along
the minor axis. Also the absorption is located along the minor and major axes.
\\

\hline

Muzahid et al. 2015: \\
\citet{2015ApJ...811..132M} \\
\citep{2015ApJ...811..132M} \\
found an outflow along the minor axis of a star-forming galaxy at $z$ = 0.4,
using QSO absorption lines. \\

\hline

Nielsen et al. 2015: \\
\citet{2015ApJ...812...83N} \\
\citep{2015ApJ...812...83N} \\
MAGIICAT paper that studies Mg \textsc{ii} properties depending on azimuthal
angle. Found that absorbers with largest velocity dispersions are located along
the minor axis. Column densities are not dependent on azimuthal angle. \\

\hline

Peek et al. 2015: \\
\citet{2015ApJ...813....7P} \\
\citep{2015ApJ...813....7P} \\
found reddening of the CGM along the minor axis at small impact parameters. \\

\hline

Schroetter et al. 2015: \\
\citet{2015ApJ...804...83S} \\
\citep{2015ApJ...804...83S} \\
observations of 10 QSO-galaxy pairs. Model that assumes a biconical outflow
along the minor axis is able to explain the observations. \\

\hline

Bouché et al. 2016: \\
\citet{2016ApJ...820..121B} \\
\citep{2016ApJ...820..121B} \\
found evidence for an inflow along the minor axis of a galaxy using MUSE
data. \\

\hline

Huang et al. 2016: \\
\citet{2016MNRAS.455.1713H} \\
\citep{2016MNRAS.455.1713H} \\
studied the CGM of 37621 LRGs at $z\sim$0.4-0.7 using Mg \textsc{ii} absorption
lines on QSOs. Found enhanced Mg \textsc{ii} absorption along the major axis of
[O \textsc{ii}] emitting LRGs at small impact parameters. \\

\hline

Péroux et al. 2016: \\
\citet{2016MNRAS.457..903P} \\
\citep{2016MNRAS.457..903P} \\
studied absorbers at $z$=0.7-1. Expected to find higher metallicity on absorbers
located along the minor axis of their host galaxy, but their results did not
show that trend. \\

\hline

Schroetter et al. 2016: \\
\citet{2016ApJ...833...39S} \\
\citep{2016ApJ...833...39S} \\
found evidence for a biconical outflow in a single galaxy using MUSE and
UVES. \\

\hline

Wotta et al. 2016: \\
\citet{2016ApJ...831...95W} \\
\citep{2016ApJ...831...95W} \\
studied 44 pLLSs and 11 LLSs at 0.1 < $z$ < 1.1. Found that metal-poor absorbers
are located along the major axis of their host galaxy, while metal-rich
absorbers are located along the minor axis. \\

\hline

Bouché 2017: \\
\citet{2017ASSL..430..355B} \\
\citep{2017ASSL..430..355B} \\
review. Relevant section: 2.1. \\

\hline

Chen 2017: \\
\citet{2017ASSL..434..291C} \\
\citep{2017ASSL..434..291C} \\
review. Relevant section: 2. \\

\hline

Faucher-Giguère 2017: \\
\citet{2017ASSL..430..271F} \\
\citep{2017ASSL..430..271F} \\
review. Relvant section: 2.4. \\

\hline

Heckman et al. 2017: \\
\citet{2017ApJ...846..151H} \\
\citep{2017ApJ...846..151H} \\
used quasars to study the CGM of 17 low-redshift galaxies. Found no correlation
between equivalent width and azimuthal angle, but attributed this to the large
impact parameters of the absorbers. \\

\hline

Kacprzak 2017: \\
\citet{2017ASSL..430..145K} \\
\citep{2017ASSL..430..145K} \\
review. Relevant sections: 2.2, 4. \\

\hline

Lehner 2017: \\
\citet{2017ASSL..430..117L} \\
\citep{2017ASSL..430..117L} \\
review. Relevant section: 9. \\

\hline

Nielsen et al. 2017: \\
\citet{2017ApJ...834..148N} \\
\citep{2017ApJ...834..148N} \\
continuation of \citet{2015ApJ...815...22K}. Found no difference in the
kinematics of O \textsc{vi} located along the major or minor axes. They conclude
that gas at intermediate azimuthal angles might be in a different ionization
state. \\

\hline

Péroux et al. 2017: \\
\citet{2017MNRAS.464.2053P} \\
\citep{2017MNRAS.464.2053P} \\
used MUSE observations to find evidence for an inflow on a single galaxy at
$z$ = 0.43. \\

\hline

Lan and Mo 2018: \\
\citet{2018ApJ...866...36L} \\
\citep{2018ApJ...866...36L} \\
studied 200000 ELGs using Mg \textsc{ii} and Fe \textsc{ii} absorption lines on
the spectra of background quasars. Found that metal absorption is two times
stronger along the minor axis than the major axis. \\

\hline

Rahmani et al. 2018: \\
\citet{2018MNRAS.474..254R} \\
\citep{2018MNRAS.474..254R} \\
used MUSE observations to study a LLS at $z$ = 0.38. Found evidence for an
inflow located around the galaxy's minor axis. \\

\hline

Rubin et al. 2018b: \\
\citet{2018ApJ...868..142R} \\
\citep{2018ApJ...868..142R} \\
used galaxies to study the CGM of 27 foreground galaxies at 0.35 < $z$ < 0.8.
Calculated a fiducial model for the Mg \textsc{ii} strength, bit did not include
azimuthal angle on it, since they could not find a dependence between line
strength and azimuthal angle. \\

\hline

Hani et al. 2019: \\
\citet{2019MNRAS.488..135H} \\
\citep{2019MNRAS.488..135H} \\
studied O \textsc{vi} absorption on 28 galaxies at $z$ = 0. Found no azimuthal
dependence of O \textsc{vi} absorption.\\

\hline

Kacprzak et al. 2019: \\
\citet{2019ApJ...886...91K} \\
\citep{2019ApJ...886...91K} \\
studied the metallicity of 25 absorption systems at $\left< z\right>$ = 0.28.
Found no correlation between metallicity and azimuthal angle. \\

\hline

Martin et al. 2019: \\
\citet{2019ApJ...878...84M} \\
\citep{2019ApJ...878...84M} \\
studied 50 galaxy-QSO pairs at $z\sim$0.2. Confirmed previous results that
Mg \textsc{ii} absorption is stronger along the minor axis. \\

\hline

Nelson et al. 2019: \\
\citet{2019MNRAS.490.3234N} \\
\citep{2019MNRAS.490.3234N} \\
simulations of outflows. Initially outflows have a wide apperture angle, but
then become collimated. \\

\hline

Ng et al. 2019: \\
\citet{2019ApJ...886...66N} \\
\citep{2019ApJ...886...66N} \\
studied 31 O \textsc{vi} absorbers at 0.12 < $z$ < 0.66. Found no dependence
between O \textsc{vi} kinematics and azimuthal angle, but absorbers located
along the minor axis had larger optical depths. \\

\hline

Pointon et al. 2019: \\
\citet{2019ApJ...883...78P} \\
\citep{2019ApJ...883...78P} \\
studied the CGM of 47 galaxies at $z$ < 0.7. Found that the CGM metallicity does
not depend on azimuthal angle. \\

\hline

Schroetter et al. 2019: \\
\citet{2019MNRAS.490.4368S} \\
\citep{2019MNRAS.490.4368S} \\
studied 79 strong Mg \textsc{ii} absorbers. Found bimodal distribution of
Mg \textsc{ii} equivalent width. \\

\hline

Zabl et al. 2019: \\
\citet{2019MNRAS.485.1961Z} \\
\citep{2019MNRAS.485.1961Z} \\
used MUSE and UVES observations to study 79 Mg \textsc{ii} absorbers at
$z\sim$1. Found a clear bimodality of the absorbers' azimuthal angle
distribution. \\

\hline

Ho and Martin 2020: \\
\citet{2020ApJ...888...14H} \\
\citep{2020ApJ...888...14H} \\
studied inflows on star-forming galaxies. Concluded that absorbers with
azimuthal angles close to 0 are not the best to probe inflows, since their
velocity components would be blended with those of the disk galaxy. \\

\hline

Zabl et al. 2020: \\
\citet{2020MNRAS.492.4576Z} \\
\citep{2020MNRAS.492.4576Z} \\
found evidence for an outflow on a single galaxy at $z$ = 0.7. They were able to
probe opposite sides of the galaxy. The outflow was located along the minor
axis. \\

\hline

Lundgren et al. 2021: \\
\citet{2021ApJ...913...50L} \\
\citep{2021ApJ...913...50L} \\
studied 54 Mg \textsc{ii} absorbers at 0.64 < $z$ < 1.4. Found enhanced
Mg \textsc{ii} absorption within 50º of the minor axis. Also calculated a model
for equivalent width distribution that takes into consideration azimuthal angle,
impact parameter and environment.
